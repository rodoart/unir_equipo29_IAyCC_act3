\documentclass[10pt,twocolumn,a4paper]{articuloAPA}
%------------------------------------------------------------------------------%
% INFORMACIÓN DEL ARTÍCULO Y METADATOS                                         %
%------------------------------------------------------------------------------%
\author{Jorge Castañeda, Carlos Alfaro, Rodolfo González}
\course{Inteligencia Artificial y Computación Cognitiva}
\activity{Prototipado de un modelo hipotético de sistema artificial basado en una función cognitiva humana}
\assignment{3}
\title{Arquitectura de un sistema evolutivo}%Provisional
\keywords{ia, unir}
\date{\today}
%------------------------------------------------------------------------------%


%%%%%%%%%%%%%%%%%%%%%%%%%%%%%%%%%%%%%%%%%%%%%%%%%%%%%%%%%%%%%%%%%%%%%%%%%%%%%%%%
\begin{document}
%------------------------------------------------------------------------------%
% Página de Título                                                             %
%------------------------------------------------------------------------------%
\twocolumn[
  \begin{@twocolumnfalse}
    \maketitle

    \begin{center}
      Basado en \fullcite{alam2020genetic}
    \end{center}

%    \begin{abstract}
%       \noindent \textit{ }
%   \end{abstract}
  %\vspace{2\baselineskip}
  \vspace{-5pt}
  \rule{\textwidth}{2pt}\vspace{7pt}
  \end{@twocolumnfalse}
]

%------------------------------------------------------------------------------%
\indexed{section}{Introducción}
%------------------------------------------------------------------------------%
El algoritmo genético o GA por sus siglas en ingles (\textit{Genetic Algorithm}) es un paradigma de aprendizaje automático que genera los patrones de comportamiento a partir de la representación de los mecanismos de evolución. Este algoritmo se define como una metaheurística motivada por el proceso de evolución y pertenece a la gran clase de algoritmos evolutivos en informática y matemáticas computacionales.

Algunos de los sectores en donde está creciendo los algoritmos genéticos son: \textit{Machine Learning}, procesamiento de imágenes, generación de rutas, optimización de problemas, optimizaciones multimodal, economía, redes neuronales, paralelización, calendarización de aplicaciones, formación robótica, arquitectura de aeronaves y bioinformática.

En el uso de algoritmos genéticos lo primero que debe hacerse es definir correctamente el problema a resolver, una vez definido puede abstraerse en los siguientes conceptos:

\begin{itemize}
  \item \textbf{Cromosoma}: contiene una posible solución, sus valores pueden ser binarios, decimales, reales, cadena de caracteres.
  \item \textbf{Alelo}: es una unidad del valor del cromosoma, en el caso del código binario sería un ``1'' o un ``0''.
  \item \textbf{Gen}: Es un conjunto de alelos que se separa para poder hacer las operaciones de reproducción y mutación.
  \item \textbf{Población}: Es un conjunto de cromosomas.
  \item \textbf{Función de aptitud}: La función de aptitud que es una función que determina la validez de las soluciones propuestas.
\end{itemize} 

%------------------------------------------------------------------------------%
\indexed{section}{Algoritmo genético}
%------------------------------------------------------------------------------%

Una vez abstraído los conceptos podemos generar el algoritmo con los siguientes pasos:

\begin{enumerate}
  \item Considere una población de P individuos al azar.
  \item La población inicial se evalúa con la función de aptitud, de esta manera podemos conservar a los candidatos con mejor punteo.
  \item 	Una vez seleccionados los candidatos procedemos a cruzarlos para obtener nuevos miembros de la población que ocupen los lugares de aquellos menos aptos.
  \item Podemos adicionar una mutación aleatoria en cada iteración para evitar que el algoritmo se cicle.
  \item Repetimos los pasos del 2 al 4 hasta que obtengamos un individuo con la aptitud deseada o lleguemos al número límite de iteraciones.
\end{enumerate}

%------------------------------------------------------------------------------%
\twocolumn[\begin{@twocolumnfalse}
  \addcontentsline{toc}{section}{Referencias}
  \printbibliography
\end{@twocolumnfalse}]
\end{document}
%%%%%%%%%%%%%%%%%%%%%%%%%%%%%%%%%%%%%%%%%%%%%%%%%%%%%%%%%%%%%%%%%%%%%%%%%%%%%%%%